\titulo{Alvorada: Protocolo baseado em Blockchain para Elaboração e Auditoria de Contratos entre Provedores de Microsserviços e Usuários}
\autor{Wilton Jaciel Loch}
\nome{Wilton Jaciel}
\ultimonome{Loch}


\bacharelado \curso{Ciência da Computação}
\ano{2019}
\data {\today}
\cidade{Joinville}

\instituicao{Universidade do Estado de Santa Catarina}
\sigla{UDESC}
\unidadeacademica{Centro de Ciências Tecnológicas}

\orientador{Prof. Guilherme Piêgas Koslovski}
\examinadorum{Prof. Charles Christian Miers}
\examinadordois{Prof. Maurício Aronne Pillon}

\ttorientador{Doutor}
\ttexaminadorum{Doutor}
\ttexaminadordois{Doutor}

\newpage
\pagestyle{empty}

\maketitle

%\begin{dedicatoria}
%\noindent
%\end{dedicatoria}
%\noindent
%\newpage
%\begin{epigrafe}
%\noindent
%\end{epigrafe}

%\agradecimento{Agradecimentos}
%caneta

\resumo{Resumo}
\noindent Os microsserviços são um modelo arquitetural cada vez mais utilizado no meio computacional. Suas vantagens, principalmente relacionadas à eficiência e escalabilidade quando comparado com o modelo monolítico, fazem com que seu uso seja visado por diversas empresas.
%
Contudo, a ampla adoção para implementação de serviços complexos e distribuídos requer a manipulação de requisições em larga escala.
%
Assim, torna-se necessária a seleção de provedores para alocar a execução dos microsserviços, bem como escalonar suas tarefas, de modo que atuem corretamente no âmbito das aplicações.
%
Um fator agravante no cenário em questão é a existência de múltiplos provedores, arquiteturalmente posicionados nas nuvens ou nas bordas da Internet, com diferentes preços, estratégias e qualidades nos seus serviços.
%
Tais provedores compõem um mercado heterogêneo e competitivo no qual adotam diferentes características do serviço.
%
Nesse ambiente, os contratos se estabelecem principalmente através de modelos pré-estabelecidos e linguagens de definição de infraestrutura disponibilizadas pelos provedores aos clientes, sobre as quais podem haver mudanças e discussões futuras, para que seja possível atingir um acordo suficientemente aceitável para ambas as partes.
%
Diante dos fatos, o presente trabalho de conclusão de curso apresenta Alvorada, um protocolo para elaboração e auditoria de contratos, formalizados entre provedores e usuários.
%
O protocolo é baseado em \textit{Blockchain}, explorando o armazenamento distribuído e verificável da arquitetura da mesma forma que seu processamento descentralizado.

\noindent \textbf{Palavras-chave:} Blockchain. IaaS. Auditoria. Confiabilidade. Consenso. SLA.

\resumo{Abstract}
 

\noindent The microservices are an architectural model increasingly used in the computational environment. Its advantages, mainly related to the efficiency and scalability when compared to the monolithic architecture, justifies the use by several companies.
%
However, the widespread adoption of complex and distributed services requires a large-scale requisitions handling.
%
Thus, it is necessary to select providers to allocate the execution of the microservices, as well as stagger their tasks, in a way that they act correctly in the scope of the applications.
%
An aggravating factor in the scenario is the existence of multiple providers, architecturally positioned in the cloud or on the edges of the Internet, with different prices, strategies and qualities in their services.
%
These providers form an heterogeneous and competitive market where they adopt different views of the service.
%
In this environment, contracts are established primarily through pre-established templates and infrastructure definition languages made available by providers to customers, on which there may be future changes and discussions, so that a acceptable agreement can be reached for both parts.
%
On this scenario, the present work proposed Alvorada, a protocol for elaboration and verification of contracts formalized between providers and users.
%
The protocol is based on the Blockchain technology, exploiting the distributed and verifiable storage of the architecture, as well as its decentralized processing aspects.

\noindent \textbf{Keywords:} \textit{Blockchain. IaaS. Trustability. Verification. Consensus. SLA.}

\tableofcontents
\listoffigures
% \listoftables
\newpage
\chapter*{Lista de Abreviaturas\hfill} \addcontentsline{toc}{chapter}{Lista de Abreviaturas}
% \listofsymbols
\begin{acronym}[ECDSA]       
    \acro{API}{\textit{Application Programming Interface}}
    \acro{CDN}{\textit{Content Delivery Network}}
    \acro{CS}{Confirmação de Serviço}
    \acro{CV}{Convocação de Votação}
    \acro{AC}{Atualização de Certificado}
    \acro{DPoS}{\textit{Delegated Proof of Stake}}
    \acro{ECDSA}{\textit{Elliptic Curve Digital Signature Algorithm}}
    \acro{ECC}{\textit{Elliptic Curve Cryptography}}
    \acro{PC}{\textit{Publicação de Certificado}}
    \acro{EOA}{\textit{Externally Owned Account}}
    \acro{EVM}{\textit{Ethereum Virtual Machine}}
    \acro{FaaS}{\textit{Function as a Service}}
    \acro{IaaS}{\textit{Infrastructure as a Service}}
    \acro{IasC}{\textit{Infrastructure as Code}}
    \acro{IoT}{\textit{Internet of Things}}
    \acro{ISP}{\textit{Internet Service Provider}}
    \acro{P2P}{\textit{Peer-to-Peer}}
    \acro{PaaS}{\textit{Plataform as a Service}}
    \acro{pBFT}{\textit{Practical Byzantine Fault Tolerance}}
    \acro{PoB}{\textit{Proof of Burn}}
    \acro{PoC}{\textit{Proof of Capacity}}
    \acro{PoS}{\textit{Proof of Stake}}
    \acro{PoW}{\textit{Proof of Work}}
    \acro{PR}{Prolongação de Requisição}
    \acro{PS}{Proposta de Serviço}
    \acro{RS}{Requisição de Serviço}
    \acro{SaaS}{\textit{Software as a Service}}
    \acro{SCaaS}{\textit{Small Cell as a Service}}
    \acro{SC}{\textit{Smart Contract}}
    \acro{SLA}{\textit{Service Level Agreement}}
    \acro{SLI}{\textit{Service Level Indicator}}
    \acro{SLO}{\textit{Service Level Objective}}
    \acro{STXO}{\textit{Spent Transaction Output}}
    \acro{UTXO}{\textit{Unspent Transaction Output}}
    \acro{VT}{Veredito e Transferências}
\end{acronym}


\newpage
\pagestyle{myheadings}
