\chapter{Conclusão}
\label{ch:conclusao}

Um problema inerente aos contratos de serviços relacionados às nuvens computacionais para microsserviços, principalmente quando fornecidos a usuários nas bordas da Internet que não possuem controle sobre dados relacionados ao serviço \cite{nuvem_sla:edge_computing}, trata-se da falta de garantia de cumprimento do acordado por parte do provedor.

%
Por meio da proposta apresentada no Capítulo \ref{ch:proposta} intende-se resolver o problema da inexistência de um ambiente concreto para elaboração de contratos \ac{IaaS} para microsserviços e da impossibilidade de garantir a veracidade de acusações de violação ou provar a violação em casos verdadeiros. A solução proposta utiliza primordialmente a tecnologia de \textit{Blockchain} para criar um protocolo que permita a elaboração e auditoria dos contratos em questão.

Foram propostas transações distintas que são utilizadas para que usuários consigam publicar informações na rede e através destas realizar os processos como a contratação e auditoria dos serviços. O protocolo confere também a cada usuário um valor em pontos de reputação que reflete de maneira quantitativa a visão da rede sobre seu comportamento ou a qualidade de seu serviço. Processos de monitoração são utilizados para gerar dados comprobatórios secretos a respeito de possíveis quebras em contratos, sendo estes verificados através de votações baseadas tanto em pontos focais quanto em dados de contextos supostamente semelhantes. O uso das votações permite a constatação de vereditos sobre culpados e inocentes, bem como a possibilidade de redistribuição da reputação, feita através de cálculos baseados na semelhança dos dados. Acredita-se que sistemas baseados em reputação como o do protocolo proposto possam ser utilizados no mundo real para os mais diversos fins na garantia de qualidade do serviço e escolha de novos parceiros de negócio.

Por fim, é relacionada uma análise teórica de todas as possíveis ameaças identificadas, suas probabilidades de acontecimento, custo de realização e possíveis danos que seriam desferidos aos usuários ou a rede como um todo. A partir desse estudo foi possível verificar que a proposta é bastante resiliente e robusta em face das diferentes ameaças apresentadas, lidando bem com a possibilidade de existência de usuários desonestos ou mal intencionados.