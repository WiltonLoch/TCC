\chapter{Ameaças ao Protocolo}
\label{ch:ameaças}

Este capítulo tem por objetivo apresentar uma visão descritiva e argumentativa sobre um conjunto de possíveis ameaças apresentadas contra o funcionamento do protocolo e como a arquitetura do mesmo é estabelecida para mitigá-las da melhor forma possível. No ramo da segurança tecnológica e de algoritmos para esse fim, não existe o conceito de completude ou inexistência de vulnerabilidades e assim, mesmo que dada solução apresente um nível alto de segurança, esta não isenta-se do surgimento de novos ataques efetivos que exploram vulnerabilidades antes não conhecidas. O mesmo naturalmente recai sobre o presente protocolo e embora um estudo refinado das possibilidades de ataques seja empregado neste capítulo, novas vulnerabilidades podem ser descobertas a qualquer momento e implicar em atualizações estruturais de modo a aumentar a robustez da proposta.

Este capítulo está dividido de forma que na Seção \ref{sec:premissas_requisitos} são salientadas todas as características relacionadas ao protocolo que fogem ao escopo da proposta e que embora possam influir certa importância no funcionamento desta, não constituem responsabilidades da mesma tratar suas definições aprofundadas. Na Seção \ref{sec:ameacas_consenso} são indicadas possíveis ameaças ao consenso da rede. 
Na Seção~\ref{sec:ameacas_gerais} são apresentadas as ameaças gerais que podem originadas por qualquer participante.
Especificamente, na Seção \ref{sec:ameacas_clientes} são apresentadas todas as ameaças levantadas que podem ser provenientes de usuários clientes e de forma semelhante na Seção \ref{sec:ameacas_prestadores} as ameaças oriundas de prestadores de serviço. Por fim, na Seção \ref{sec:consideracoes_ameacas} tratam-se as considerações finais à respeito das ameaças e do capítulo em si.

% TODO: a ordem das seções está incorreta.

\section{Premissas e Requisitos}
\label{sec:premissas_requisitos}

Esta seção tem por objetivo tratar as características do protocolo que não estão contidas em seu escopo de definição ou desenvolvimento. O primeiro aspecto importante é a representação das informações referentes às infraestruturas que são requeridas pelos usuários, nessa planificação teórica da proposta não são estipuladas linguagens específicas para a definição ou estruturas de dados para representar as infraestruturas ou funções que desejam ser contratadas. A partir de tal princípio, o protocolo não fica fortemente dependente da forma como os dados são representados e futuras implementações podem fazer uso de diferentes tecnologias de estruturação que melhor atendam aos requisitos de prestadores e clientes no mundo real. O protocolo também torna-se mais facilmente atualizável, de modo a comportar novas tecnologias ou características de infraestruturas antes não utilizadas.

Outro fator isento do escopo do protocolo é a conferência do certificado inicial, como se dá o processo manual de conferência da identidade dos prestadores de serviço ou a definição de qual seria a entidade certificadora responsável pelo mesmo. Acredita-se entretanto, que a forma mais sensata de distribuição dos certificados seria feita através da própria organização ou grupo responsável pela manutenção do protocolo. Assim, da mesma forma que em redes como Bitcoin e Ethereum há desenvolvedores dedicados à criação de atualizações e correção de vulnerabilidades, a tarefa de certificação poderia também estar atrelada às atividades dos indivíduos que desempenhariam esse papel na presente proposta. A certificação, como já mencionado na Seção \ref{sec:proposta:definicoes}, seria o único ponto de centralização da rede, porém não afeta de forma alguma a própria forma descentralizada do protocolo, uma vez que a ausência da entidade em um certo ponto no tempo apenas preveniria a entrada de novos prestadores de serviço, não afetando no funcionamento da rede em si e atividades de estabelecimento de contratos e auditorias.

O protocolo também não possui como responsabilidade o controle de funcionamento das alocações de infraestruturas ou funções realizadas internamente por prestadores de serviço, bem como não é seu objetivo criar sistemas de suporte ou redes que tratem especificamente da comunicação com as infraestruturas ou acesso a outros serviços de seus provedores responsáveis. Portanto, a proposta visa estabelecer um ambiente de auxílio no estabelecimento dos contratos e suas auditorias porém não toma nenhuma forma de interação com a implantação dos serviços em si. A mesma lógica aplica-se também para a inserção do módulo de monitoração nas infraestruturas contratadas. 

Por fim, o protocolo não compreende a definição ou implementação do módulo de monitoração, embora faça uso extensivo das funcionalidades do mesmo. Dessa forma, na Seção \ref{sec:proposta:monitoracao} é feita apenas uma descrição de alto nível do que é esperado do módulo em termos de dados de retorno, entrada e realização dos testes, porém as estratégias utilizadas para monitoração, estruturas de dados internas, formas de armazenamento, segurança das chaves e outras características inerentes de implementação não são contempladas por esse trabalho. Contudo, é considerada necessidade pétrea independente de qualquer implementação a alta confidencialidade e segurança da chave privada do módulo, localizada ou construída em seu interior. 

\section{Formas de Consenso e Ameaças}
\label{sec:ameacas_consenso}

A designação de uma forma de consenso definitiva ou ideal para o protocolo não é buscada, primeiramente pois o próprio conceito de um consenso idealizado para o presente contexto pode não ser encontrado em meio as diversas opções. Em segundo lugar, a necessidade de concretização de uma forma de consenso, embora passível de pequenas influências em características secundárias do funcionamento do protocolo, é vista apenas como um aspecto atrelado a implementação do mesmo. Visto que se as características básicas de operação do Blockchain forem mantidas, o protocolo tem êxito independentemente da forma de consenso subjacente, dada é claro a funcionalidade da mesma.

%
Uma característica costumeiramente relacionada ao \ac{PoW} é a baixa vazão de transações, criada pela própria dificuldade dos blocos e pela espera recomendável de seis blocos posteriores para garantir confirmação. Essa dita desvantagem não se apresentaria como um obstáculo à proposta uma vez que não são necessários tempos curtos entre transações subsequentes. Contudo, a principal ameaça do \ac{PoW} para o contexto do protocolo - já mencionada na Seção \ref{sec:proposta:definicoes} - é a abismal diferença de poder computacional entre clientes e prestadores de serviço, que apresenta indícios de centralização e possibilidades mais altas de ataques. Quanto a escalabilidade, não é esperado que a presente rede eventualmente implementada e popularizada atinja a mesma quantidade de nós presente em redes públicas, especificamente criptomoedas de propósito geral como Bitcoin e Ethereum. Afirma-se essa característica pois estima-se que os grupos de prestadores de serviço e clientes possuam centenas e milhares de usuários respectivamente, em prospectos de alta popularidade do protocolo. Portanto, especula-se que a mais coerente forma de consenso seria derivada da tolerância a falhas bizantinas, utilizando algoritmos como o \ac{pBFT}, Tendermint ou outros.

%
O conjunto de ameaças conferidas ao protocolo devido a forma de consenso são naturalmente apenas ameaças intrínsecas às limitações do consenso em si. Portanto, ao utilizar sistemas baseados em falhas bizantinas tolera-se até $1/3$ de nós desonestos, ao utilizar \ac{PoW} até $50\%$ do poder computacional e assim por diante.

\section{Ameaças Gerais}
\label{sec:ameacas_gerais}

Ameaças gerais consistem de um conjunto de possíveis ações que seriam prejudiciais ao protocolo e que teoricamente podem ser efetuadas por qualquer usuário, independente de seu grupo de participação. A primeira e mais facilmente identificável ameaça é a falsificação ou fraude de transações, pois como estas não passam de uma sequência ordenada de dados específicos, basta que um participante encontre uma maneira de criar transações sem utilizar a carteira, replicando assim os padrões de organização dos dados, para que este tenha poder total de gerar transações que normalmente não possuiria permissões ou transações com dados incoerentes ou incorretos. Essa ameaça é felizmente tratada pelo próprio funcionamento do Blockchain em si, uma vez que qualquer usuário pode escrever qualquer transação que não tenha informações verdadeiras e tentar repassá-la, porém quando outro usuário que seja honesto recebê-la e executar o algoritmo de verificação 
concluirá que a transação é inválida e não a repassará para os próximos nós da rede, evitando assim a possibilidade de que uma transação fraudulenta seja inserida na cadeia. A única possível situação em que uma transação inválida poderia ser inserida na cadeia é se a característica de comprovação da forma de consenso (\textit{e.g.}, poder computacional, votos, \textit{etc.}) fosse controlada por uma maioria de nós desonestos em concordância de um mesmo objetivo ilegítimo.

%
Ainda no mesmo espectro de ameaças encontra-se a tentativa de utilização de chaves ou certificados falsos, de forma que o usuário anexa tais informações falsas em uma transação, porém novamente a simples verificação pelo nó recebedor já é suficiente para conter tais ataques. Uma possível ameaça mais difícil de ser enfrentada é a descoberta de chaves por usuários desonestos, que não pode ser facilmente contornada e que pode desdobrar-se em ataques bastante nocivos com a descoberta da chave privada do módulo, entidade certificadora ou através da própria falsificação bem sucedida de identidade de outros usuários. Entretanto essa não pode ser considerada uma vulnerabilidade do protocolo pois a proteção eficaz de chaves privadas é uma tarefa inerente aos seus donos e a segurança contra um possível processo de criptoanálise que venha a resultar no descobrimento de chaves depende do nível de robustez do método de criptografia empregado, que no caso de \ac{ECC} é bastante alto. 

A fase de estabelecimento de contratos não possui um conjunto de vulnerabilidades nocivas, podendo apenas ser alvo de tentativas de falsificação, que como já mencionado nessa seção são ataques facilmente contornados. Isto porque todas as transações dessa fase apenas relacionam dados claros necessários para os contratos e possuem uma ordem de referenciação bastante simples, o que torna a verificação e consequentemente a garantia de segurança do processo muito mais simples já que todas as informações podem ser atestadas ou rejeitadas no processo de validação em si. Contudo, ainda nesse aspecto, um campo em particular poderia apresentar problemas e este é a semente criptografada que deve ser apresentada por prestadores de serviço em qualquer proposta, justamente pela incapacidade da validação direta do campo. Assim, nada impediria um prestador de ao invés de seguir o processo correto - escolhendo um número aleatório e o criptografando com a chave pública do módulo - simplesmente publicasse na transação um valor claro qualquer. Esse inconveniente não poderia ser resolvido com assinaturas por exemplo, uma vez que seria necessário a chave privada correspondente para verificá-las e essa é secreta e apenas de posse do módulo. Entretanto, esta não trata-se de fato de uma vulnerabilidade e não prejudica operacionalmente o protocolo, isso por que mesmo que seja publicado um valor qualquer não criptografado, ao ser inserido no módulo o algoritmo de descriptografia será aplicado e irá gerar um retorno de erro, devido a inconsistência dos dados informados para o modelo de criptografia em questão. Caso uma situação dessa venha a ocorrer o módulo pode utilizar o próprio valor não criptografado enviado para o processo de formação da chave. Entretanto a chave pública resultante não será retornada caso o terceiro campo esteja presente - como elucidado na Seção \ref{fig:votacao_detalhada} - e seja igual ao valor inválido. Assim obtêm-se um novo valor qualquer para a utilização como semente a partir de qualquer outro valor inserido inicialmente.

O único efeito negativo proveniente de tal ação seria que o provedor não seria capaz de obter a chave pública de votação que possibilita a visualização dos dados dos usuários em uma votação, caso uma viesse a ocorrer. Assim, supõe-se que não haveria nenhuma vantagem real na realização do envio incorreto do que deveria ser a semente criptografada, pois o prestador de serviço apenas estaria abdicando do poder de visualizar os dados a troco de nenhuma recompensa aparente, enquanto o usuário acusador ainda o possuiria. O mesmo raciocínio naturalmente se aplica às sementes fornecidas pelos clientes nas convocações de votação.

Ao tratar da fase de auditoria, existe um conjunto relativamente maior de possíveis ameaças ao funcionamento do protocolo devido principalmente ao uso de criptografia, sensibilidade dos dados e interações mais restritivas entre os participantes. Ao retomar o conceito de pontos focais para a obtenção de conclusões coletivas, não deve ser permitida a troca de informações entre os participantes, pois a comunicação entre estes permitiria o surgimento de conluios cujos participantes enviariam respostas imprecisas ou incorretas propositalmente, de modo a aumentar suas recompensas ou evitar punições.

%
O mesmo princípio deve ser aplicado às votações utilizadas na proposta que fundamentalmente baseiam-se nesse conceito. Entretanto, o protocolo não possui a capacidade de evitar a comunicação deliberada de dados de votação entre os participantes, isso porque embora os dados sejam criptografados de diferentes formas em períodos distintos da votação, deve haver inevitavelmente uma forma de visualizá-los para que seja possível gerar o veredito coletivo apropriado. Atrelado a este fator, como o módulo de monitoração responsável por gerar e armazenar tais dados funciona de maneira \textit{standalone} e assíncrona, a única porta de saída de informações existente é através do próprio usuário dono que as requisita. Não havendo garantias da idoneidade das intenções deste usuário quanto aos seus dados, esse pode seguir os passos descritos na Seção \ref{fig:votacao_detalhada} para iniciar uma votação, que o módulo naturalmente assume ser honesta. Contudo, ao obter os dados criptografados e a chave pública que os revela, o usuário não inicia de fato uma votação na cadeia, visualiza seus próprios dados e os compartilha com outros participantes ou os utiliza de outra forma não permitida.

%
A partir dessa análise, usuários desonestos podem de forma bastante simples ter acesso aos seus próprios dados e dar diversos fins para essa informação, porém não é possível que um usuário desonesto tenha de alguma maneira acesso a dados de usuários terceiros honestos no que toca a segurança do protocolo em si. Tal constatação pode ser concebida pois no caso do envio de um voto honesto não há como visualizar os dados, uma vez que estes estão criptografados pela chave de votação, conhecida apenas pelo usuário acusador e provedor acusado. No caso do próprio usuário convocar uma votação a mesma lógica é aplicada. A divulgação dos resultados de monitoração por si só não representa um problema. Complicações relacionadas à integridade das votações originadas de revelações apenas surgiriam se os usuários recebedores de informações as utilizassem como referencial para a prática de ações desonestas.

%
Usuários podem apresentar comportamento desonesto buscando obter reputação individual ou prejudicar a reputação de algum terceiro, seja este cliente ou prestador de serviço. Para atingir tais objetivos, é necessário que o usuário tenha posse de dados de monitoração falsos que direcionem a votação aos seus interesses e para obtê-los no contexto do protocolo, duas possíveis estratégias poderiam ser empregadas. A primeira trata-se da forja de dados, que implica na construção de dados inventados que não possuem nenhuma relação com monitorações reais, assim observando os padrões de escrita do módulo, certo usuário cria um novo arquivo e povoa-o com dados que atendem aos seus objetivos seguindo a mesma formatação que seria utilizada nos resultados reais de um módulo. A segunda forma de criação de dados falsos é a emulação, que para ser realizada possui uma carga de trabalho maior que a forja e consiste em inserir o módulo de monitoração em uma infraestrutura ou máquina virtual e simular o comportamento desejado por meio de perturbações externas no momento de realização dos testes, fazendo com que os dados sejam irreais embora originados do módulo de forma autêntica. Uma forma simples de realizar emulações seria executar aplicações computacionalmente custosas no momento dos testes de forma que o módulo não tenha os recursos que normalmente lhe seriam fornecidos em condições de teste normais.

%
A forja de dados falsos é tratada de maneira bastante simples pelo protocolo, através do uso de assinaturas e criptografia em geral. Primeiramente a chave privada de votação é construída de forma secreta utilizando-se das sementes fornecidas tanto pelo provedor acusado quanto pelo usuário acusador. Os dois valores são concatenados e então criptografados com a chave privada do módulo, gerando um conjunto resultante de bits secretos que não pode ser construído em tempo polinomial por nenhum participante de forma exterior ao módulo. Portanto, mesmo que um usuário copie o formato de escrita e crie dados fantasiosos, não seria possível criptografá-los com a chave de votação correta, pois este não a possui e também não é capaz de gerá-la. Caso certo usuário descubra a chave privada de uma votação, a mesma torna-se comprometida em toda a sua duração.

%
O uso de emulação para a criação de dados falsos possui remediação menos trivial, pois não há indícios facilmente verificáveis que indiquem a realização de um ataque desse gênero. O protocolo então dispõe de mecanismos que tornam amenas as possibilidades de realização de tais emulações ao invés de métodos de remediação e verificação de ocorrência. As emulações são aqui divididas em dois tipos principais de acordo com a forma de realização para simplificar a análise das medidas de segurança empregadas.

\begin{itemize}
    \item \textbf{Emulação Ativa:} A primeira forma de emulação é a chamada emulação ativa, que acontece quando um usuário indiscriminadamente inicia um processo de monitoração em uma infraestrutura adulterada sem necessariamente possuir algum dado de monitoração prévio de qualquer outro usuário.
    Os resultados desse gênero de ataque tendem a não ter como fim o benefício da própria conta ou endereço usado para a realização, isso porque a criação de dados falsos de maneira cega tende também a gerar resultados que destoam dos que serão enviados por outros usuários participantes na votação, uma vez que o contexto da infraestrutura monitorada não é o mesmo. Individualmente então, a única forma plausível de justificativa para tal ataque seria utilizar os dados falsos através de um segundo endereço que propositalmente perderia a votação, enquanto o endereço principal fornecendo dados assertivos absorveria a reputação perdida pelo secundário. Outra alternativa aos ataques individuais seria a formação de coalizões de diversos usuários desonestos que entram em acordo sobre o perfil da emulação que deve ser realizada, assim se a quantidade de atacantes for grande o suficiente em comparação com o total de votantes, a média de veredito resultante ficará muito mais próxima dos dados falsos do que das monitorações honestas, sem a necessidade de contas secundárias para os usuários desonestos.
    
    \item \textbf{Emulação Reativa:} A segunda forma de emulação é intitulada emulação reativa, nessa o atacante insere o módulo em uma infraestrutura adulterada e utiliza dados de monitoração de outros usuários descobertos de alguma forma para dirigir os testes aos seus objetivos. Essa trata-se de uma alternativa mais branda de realização de emulação e pode partir de um processo prévio de monitoração honesta, de maneira que apenas uma fração dos testes no período são emulados ou dirigidos. A reatividade do processo se dá pela necessidade de haver quebra de sigilo acerca dos dados de um ou mais usuários participantes da votação, sendo esse o evento que desencadeia o inicio de um ataque desse gênero. Portanto, para que a emulação possa acontecer deve haver primeiro a revelação da chave pública de votação por parte do cliente ou prestador de serviço antes do final da mesma ou revelação de dados individuais de usuários via canais cobertos. Descobrindo a chave pública de votação antes do final do \textit{locktime}, um usuário malicioso pode utilizá-la para descriptografar todos os votos já inseridos na cadeia e utilizando o algoritmo de cálculo de veredito, estimar qual a média da votação que define o resultado até aquele momento. Possuindo a média, o usuário começa a dirigir seus testes para que seus resultados coincidam o máximo possível com ela, enviando seus dados apenas próximo ao final do processo e aumentando sua recompensa individual. No caso da obtenção de dados claros de usuários de outra forma o processo empregado é o mesmo, porém a coleta é mais trabalhosa.
\end{itemize}

A emulação ativa pode ser considerada a mais custosa, principalmente quando os dados adulterados se destinam a votações com longos períodos de monitoração, pois é necessário que o atacante disponha de uma infraestrutura por períodos de tempo longos que dependendo dos objetivos de falsificação podem também torná-la inutilizável para outros propósitos. Em contrapartida, a emulação reativa é menos custosa e tem possibilidades mais sólidas de recompensas individuais que a ativa, porém depende da revelação de chaves por outros usuários desonestos ou de métodos alternativos para a obtenção dos dados de terceiros, que como já mencionado nessa seção não trata-se de uma tarefa trivial. As Seções \ref{sec:ameacas_clientes} e \ref{sec:ameacas_prestadores} destinam-se a relacionar a praticabilidade dos ataques citados até o presente momento com o grupo de clientes e prestadores respectivamente, bem como a forma como o protocolo utiliza de mecanismos para mitigá-las.

\section{Ameaças Oriundas dos Clientes}
\label{sec:ameacas_clientes}
%
No que concerne ao cliente, teoricamente a emulação ativa individual não trata-se de um ataque vantajoso, porque para que seja possível obter a reputação em uma conta primária, é necessário também uma conta secundária, sendo que ambas devem possuir contratos ativos com o mesmo provedor. Em seguida é necessário criar os dados falsos, despendendo tempo e recursos já contratados. Finalmente, deve ser realizada uma convocação de votação a partir do endereço secundário e se os dados estiverem suficientemente diferentes do restante da rede o endereço primário recebe uma fatia da reputação.

%
A partir de uma análise inicial, a realização de tal ataque implica no dobro de custo de um comportamento honesto devido à necessidade de dois contratos semelhantes ativos, que naturalmente possuem um custo atrelado. Em seguida existe a possível subutilização de uma infraestrutura que destinada a outros fins poderia estar de alguma forma gerando lucro. Por fim, mesmo que o ataque seja bem sucedido, o endereço primário recebe apenas uma pequena fração da reputação cujo total é dividido entre os outros votantes também, sendo que estes não tiveram nenhum custo extra com seus comportamentos honestos.
%
Outra possibilidade são as emulações múltiplas, que são realizadas por um cliente de modo a ter um conjunto de outras contas falsas que suportem seus dados em uma acusação garantindo sua vitória. Novamente, tal ataque é considerado demasiadamente custoso para execução pois o realizador deveria possuir um conjunto de infraestruturas contratadas possivelmente inutilizadas destinadas apenas para a realização do ataque.
%
Assim, estima-se que a realização de emulação ativa por parte de clientes não seja recompensante nesses e em outros cenários devido aos custos consideravelmente altos em comparação com os benefícios obtidos.

As emulações ativas em coletivo são menos custosas e tecnicamente menos complexas de serem realizadas. O maior desafio para a realização dessa forma de ataque é reunir um conjunto satisfatoriamente grande de usuários que possuem o mesmo contrato para que se possa influenciar a votação, pois caso o grupo não seja numeroso o suficiente os usuários ao invés de receberem recompensas pelo ataque terão desconto em reputação, pois mesmo em conluio os dados falsos ainda estariam muito distantes da maioria correta. Novamente, graças ao conceito de pontos focais, os usuários tendem a não realizar ações desonestas se acreditarem que a maioria dos outros participantes manterá um comportamento honesto, pois espera-se primeiramente que a porcentagem majoritária dos envolvidos na rede seja de fato honesta, sendo este um pré-requisito do próprio Blockchain e do protocolo, e consequentemente porque as ações da maioria ditam a distribuição de recompensas. Logicamente, nos momentos iniciais da implantação da rede, a formação de grupos desonestos é menos complexa devido a menor quantidade de usuários na rede como um todo, porém esse problema é comum também em outras formas de consenso distribuído, até mesmo os lastreados como o próprio \ac{PoW}, pois quanto menor a quantidade de usuários em uma rede que o utiliza, mais simples torna-se possuir 51\% do poder computacional individualmente ou em conluios. Portanto, a praticabilidade dessa forma de emulação está diretamente relacionada com a quantidade de usuários e o tamanho da rede em si. 

%
No que diz respeito à emulações reativas, não há um custo atrelado muito alto, uma vez que como já explanado nessa seção um conjunto de monitorações honestas pode ser usado como base para anexação de testes emulados. As possibilidades de realização de emulações desse tipo estão atreladas principalmente a dois fatores, que são a descoberta de informações e a disponibilidade de tempo. Se o primeiro fator é desencadeado por quebras de confidencialidade decorrentes de ataques de um usuário desonesto em usuários honestos, então o protocolo não possui mecanismos de defesa, uma vez que a segurança de eventuais \textit{hosts} e serviços utilizados pelos participantes é de responsabilidade única destes. Para que a descoberta seja por meio da revelação da chave pública de votação sem que ocorram invasões, acredita-se que o usuário que a divulga, seja este o cliente ou o prestador de serviço, deve ter algum interesse ou recompensa buscada através de tal ação. Entretanto, os interesses de usuários que realizam a emulação reativa são sempre contrários aos interesses que um usuário acusador ou acusado em uma votação tem com a liberação da chave. Dessa forma, tanto o cliente quanto o prestador de serviço envolvidos em uma acusação, caso desonestos apenas teriam motivos para divulgar a chave se acreditassem que perderiam a votação. 

%
Liberar a chave contundo, apenas reforçaria o resultado já esperado, pois os usuários que realização emulações reativas para obter recompensas, aproximariam seus testes da média, que não seria favorável ao usuário que revelou a chave. Para que os interesses coincidam seria necessário que o usuário revelador da chave convencesse um grande grupo de usuários a emularem seus dados de modo a dirigir a média, porém como já apontado nessa seção, tal ação é pouco trivial. O caminho inverso, no qual um usuário que pretende emular dados convence o acusado ou acusador da votação a revelar a chave também é pouco provável, pois se este acredita que perderá, o cenário anterior acontece e se acredita que vencerá, sua recompensa seria menor graças ao maior número de participantes assertivos que implica em um divisor maior para os lucros em reputação.

O segundo fator citado, que é a disponibilidade de tempo, diz respeito justamente ao tempo que um usuário tem disponível para realizar a emulação, de forma que tendo conhecimento dos dados e um período suficientemente próximo ao apresentado nos dados de acusação - que dita a quantidade de testes que deve ser enviada nos votos - torna-se mais simples fazer com que os dados emulados coincidam exatamente com a média. Entretanto, conforme o tempo disponível para realizar as emulações diminui, a dificuldade de aproximação dos testes à media sobe de maneira proporcional. O protocolo então faz uso dessa constatação para diminuir as probabilidades de realização de emulações através do \textit{locktime} presente nas convocações de votação. Assim, se o tempo de votação for consideravelmente menor que o tempo necessário para coletar o mesmo conjunto de monitorações apresentado como prova, a gravidade de ocorrências de emulações será baixa. Assim, o tempo em que votos podem ser enviados a uma votação, determinado pelo \textit{locktime} multiplicado pelo tempo médio de criação de um bloco, deve equivaler apenas a uma fração do tempo necessário para gerar os dados da acusação.

Os clientes embora possuam uma série de possibilidades de realização de ações desonestas enfrentam sérias barreiras impostas pelo protocolo quanto a possibilidade de recompensas de ataques e os custos dos mesmos.

\section{Ameaças Oriundas dos Prestadores de Serviço}
\label{sec:ameacas_prestadores}

A discussão acerca das ameaças oferecidas pelos prestadores de serviço é bastante semelhante à apresentada para os clientes na Seção \ref{sec:ameacas_clientes}, baseando-se nas possibilidades de ataques citados na Seção \ref{sec:ameacas_gerais}. Da mesma forma que para os clientes, a forja de dados é facilmente verificada e não é considerada factível para prestadores. Todavia, no caso de emulações, há aspectos distintos no grupo dos prestadores de serviço que demandam novas análises. O primeiro é a facilidade de criar contas falsas personificando clientes, pois um cliente desonesto qualquer embora consiga criar um segundo endereço falso, não consegue criar uma conta que simule um prestador de serviço pois este não seria capaz de obter o certificado que garantiria autenticidade à sua conta falsa. Entretanto, como não há necessidade de confirmação de identidade para clientes, um provedor pode simular clientes de maneira imoderada. O segundo fator é a diminuição de custos para certos ataques, pois no caso de um cliente desonesto atentar contra uma votação é necessário que ele tenha estabelecido um contrato com o prestador acusado, naturalmente implicando em custos. No caso de um provedor desonesto que cria clientes falsos, é possível que contratos sejam estabelecidos entre ambas as contas sem gastos de ativos físicos isso porque um cliente falso não necessita da alocação de uma infraestrutura e também não há pagamentos realizados.

%
Obviamente, para que um prestador de serviço consiga realizar emulações, é necessário de qualquer forma uma infraestrutura nos moldes da suposta contratação e embora não existam custos de contratação em si, há ainda a subutilização da infraestrutura que poderia estar sendo comercializada e gerando receita para o prestador em questão. Entretanto, o simples fato de ser necessária apenas uma infraestrutura que não possui custos de contratação para emulação facilita bastante a forma individualizada de ataque citada para emulações ativas na Seção \ref{sec:ameacas_gerais}, na qual um usuário cria um segundo endereço e absorve parte de sua reputação em uma votação através do envio de dados incorretos pelo primeiro.

Essa forma de ataque embora substancialmente barata de ser concretizada por prestadores de serviço não oferece nenhum ganho para os mesmos, pois nesse caso uma segunda conta falsa representando um cliente seria utilizada para atacar o próprio endereço do provedor com dados falsos, porém como definido na Seção \ref{sec:proposta:auditoria:vt} o usuário acusado não recebe reputação caso seja declarado inocente em uma votação e assim apenas os usuários votantes assertivos receberiam os pontos provenientes do cliente. O caminho contrário através de emulação múltipla, no qual o provedor emula um conjunto de dados e personifica um grupo de clientes para defendê-lo em uma votação é considerado demasiadamente custoso em vista das perdas de reputação evitadas. Isso graças ao fato de que com o crescimento da rede e da quantidade de contratos estabelecidos, o montante de instâncias emuladas necessárias cresce de maneira proporcional, tornando o volume de recursos despendidos para ataques efetivos muito alto. 

Ainda no que toca a personificação de clientes por parte do prestador de serviço, após a criação bem sucedida de um contrato entre o cliente falso e o prestador, o segundo necessariamente informou um valor de semente criptografada legítimo em sua proposta de serviço, porém criou também uma semente ilegítima para uso na conta do cliente. Dessa forma, o prestador desonesto tem conhecimento indevido das duas sementes que são utilizadas na construção das chaves de votação. Caso o módulo utilizasse no processo apenas os valores descriptografados das sementes, o prestador poderia sem grandes dificuldades replicar o algoritmo de construção e assim obter a chave privada de votação. Ao produzir tal chave com sucesso, a forja dos dados seria possível e para tal bastaria usar a chave descoberta para criptografar e assinar quaisquer resultados de monitorações inventados. Então, uma horda de clientes falsos que não possuem nenhum custo de criação ou de monitoração atrelado poderia ser usada pelo prestador para direcionar a votação utilizando estes dados forjados. Contudo, como apontado na Seção \ref{subsec:chaves_votacao} os dados recebidos tanto pelo cliente quanto pelo prestador após serem descriptografados e concatenados são novamente criptografados com a chave privada do módulo, para que só então seja obtido o conjunto de bits equivalente a chave privada. Tal característica garante que mesmo que um usuário possua conhecimento dos valores claros de ambas as sementes, não será possível reconstruir a chave de votação, desde que a confidencialidade da chave privada do módulo seja mantida.

%
A realização de emulações ativas em coalizões assume exatamente o mesmo caráter para prestadores de serviço e clientes, no qual a discussão apresentada na Seção \ref{sec:ameacas_clientes} é válida. Assim, para que tal emulação feita por um prestador gere efeitos na rede é necessário a formação de um conluio suficientemente grande para gerar impactos significativos na média da votação. Essa associação pode ser logicamente realizada com participantes de ambos os grupos, não havendo grandes diferenças em seus poderes de adulteração das votações dados todos os gastos envolvidos na emulação. Semelhantemente, a emulação reativa se apresenta da mesma forma tanto no grupo dos clientes quanto no grupo dos prestadores de serviço. 

\section{Considerações Finais}
\label{sec:consideracoes_ameacas}

O levantamento do conjunto de possíveis ameaças apresentadas ao protocolo, seus mecanismos para tratá-las e possíveis novas vulnerabilidades é apresentado nesse capítulo de um ponto de vista teórico e amplo. Porém, a variedade de cenários e possibilidades de ações desonestas exploradas compreende significativamente os contextos de uso da proposta e seu fluxo de funcionamento, explorando interesses e esperanças de recompensa relacionadas aos ataques que poderiam ser desempenhados por usuários desonestos.

%
A tabela \ref{tab:ameacas} relaciona as principais ameaças identificadas no levantamento e indica características importantes das mesmas quanto a praticabilidade e possibilidade de danos causados a rede. As colunas título e origem indicam o nome conferido a ameaça e de qual usuário a mesma pode ser proveniente, respectivamente. A coluna chamada gravidade indica o quão danosa a concretização de uma ameaça seria à rede, na qual muito baixa indica que dificilmente algum dano seria infringido; 
%
Baixa indica que possivelmente danos pequenos poderiam ser infringidos ou a concretização de tal ameça facilitaria a realização de ameaças maiores; 
%
Média indica probabilidade de dano a um outro usuário ou grupo pequeno de usuários; 
%
Alta indica dano a um conjunto significativo de usuários ou a processos completos na rede; 
%
Muito alta indica grandes danos que podem influenciar todos os usuários da rede e até o próprio funcionamento do protocolo em si.

%
Por fim, a coluna intitulada Custo mostra o quão custosa ou complexa seria a concretização de certa ameaça, de forma que baixo compreende a realização simples, que pode ser feita sem dificuldades por um usuário desonesto e não despende custos computacionais, financeiros ou reunião de outros usuários;
%
Médio indica a existência de gasto computacional ou financeiro na concretização da ameaça; 
%
Alto indica gastos financeiros ou computacionais altos, bem como a possibilidade de reunião de pequenas coalizões; 
%
Muito alto indica a necessidade de custos computacionais ou financeiros muito elevados ou a reunião de enormes coalizões.

\begin{table}[ht]
    \centering
    \begin{tabular}{|m{0.25\textwidth}|m{0.2\textwidth}|m{0.215\textwidth}|m{0.215\textwidth}|}
    \hline
         \textbf{Título} & \textbf{Origem} & \textbf{Gravidade} & \textbf{Custo} \\
         \hline
         Controle do consenso & Qualquer usuário & Muito alta & Muito alto \\
         \hline
         Transações inválidas & Qualquer usuário & Alta & Muito alto \\
         \hline
         Descoberta de chave de outro usuário & Qualquer usuário & Média & Muito alto \\
         \hline
         Descoberta de chave do módulo & Qualquer usuário & Muito alta & Muito alto \\
         \hline
         Descoberta de chave da entidade certificadora & Qualquer usuário & Muito alta & Muito alto \\
        %  \hline
        %  Descoberta de chave da entidade certificadora & Qualquer usuário & Muito alta & Muito alto   \\
        \hline
         Publicação de dados de monitoração & Qualquer usuário & Baixa & Baixo \\
         \hline
         Envio de semente incorreta & Qualquer usuário & Muito baixa & Baixo \\
         \hline
         Não envio de veredito & Qualquer usuário & Baixa & Baixo \\
         \hline
         Forja de dados & Qualquer usuário & Alta & Muito alto \\
         \hline
         Emulação reativa & Qualquer usuário & Média & Muito alto \\
         \hline
         \multirow{2}{0.25\textwidth}{Emulação ativa única individual} & Cliente & Baixa & Alto \\
         \cline{2-4}
         & Prestador de ser. & Muito Baixa & Médio \\
         \hline
         \multirow{2}{0.25\textwidth}{Emulação ativa múltipla individual} & Cliente & Baixa & Muito Alto \\
         \cline{2-4}
         & Prestador de ser. & Muito Baixa & Alto \\
         \hline
         Emulação ativa em coalizão & Qualquer usuário & Alta & Alto \\
    \hline
    \end{tabular}
    \caption{Conjunto das ameaças encontradas}
    \label{tab:ameacas}
\end{table}



%
Portanto, acredita-se que a análise produzida reflete assertivamente a robustez do perfil de segurança apresentado pela proposta para lidar com ataques de diferentes grupos. No entanto, é comum no meio da segurança computacional a descoberta de novas vulnerabilidades e ataques que as explorem ou até mesmo possíveis ataques mais complexos não previstos na análise inicial, sugerindo assim a possibilidade de novas avaliações ou análises de outros gêneros que mantenham a confiabilidade do sistema alta.