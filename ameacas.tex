\chapter{Ameacas ao Protocolo}
\label{ch:ameaças}

Este capítulo tem por objetivo apresentar uma visão descritiva e argumentativa sobre um conjunto de possíveis ameaças apresentadas contra o funcionamento do protocolo e como a arquitetura do mesmo é estabelecida para mitigá-las da melhor forma possível. No ramo da segurança tecnológica e de algoritmos para esse fim, não existe o conceito de completude ou inexistência de vulnerabilidades e assim, mesmo que dada solução apresente um nível alto de segurança, esta não isenta-se do surgimento de novos ataques efetivos que exploram vulnerabilidades antes não conhecidas. O mesmo naturalmente recai sobre o presente protocolo e embora um estudo refinado das possibilidades de ataques seja empregado neste capítulo, novas vulnerabilidades podem ser descobertas a qualquer momento e implicar em atualizações estruturais de modo a aumentar a robustez da proposta.

Este capítulo está dividido de forma que na Seção \ref{sec:premissas_requisitos} são salientadas todas as características relacionadas ao protocolo que fogem ao escopo da proposta e que embora possam influir certa importância no funcionamento desta, não constitui responsabilidade da mesma tratar suas definições aprofundadas. Na Seção \ref{sec:ameacas_clientes} são apresentadas todas as ameaças levantadas que podem ser provenientes de usuários clientes e de forma semelhante na Seção \ref{sec:ameacas_prestadores} as ameaças oriundas de prestadores de serviço. Na Seção \ref{sec:ameacas_consenso} são indicadas possíveis ameaças ao consenso da rede e por fim, na Seção \ref{sec:consideracoes_ameacas} tratam-se as considerações finais à respeito das ameaças e do capítulo em si.

\section{Premissas e requisitos}
\label{sec:premissas_requisitos}

Esta seção tem por objetivo tratar as características do protocolo que não estão contidas em seu escopo de definição ou desenvolvimento. O primeiro aspecto importante é a representação das informações referentes às infraestruturas que são requeridas pelos usuários, nessa planificação teórica da proposta não são estipuladas linguagens específicas para a definição ou estruturas de dados para representar as infraestruturas ou funções que desejam ser contratadas. A partir de tal princípio, o protocolo não fica fortemente dependente da forma como os dados são representados e futuras implementações podem fazer uso de diferentes tecnologias de estruturação que melhor atendam aos requisitos de prestadores e clientes no mundo real. O protocolo também torna-se mais facilmente atualizável, de modo a comportar novas tecnologias ou características de infraestruturas antes não utilizadas.

Outro fator isento do escopo do protocolo é a conferência do certificado inicial, como se dá o processo manual de conferência da identidade dos prestadores de serviço ou a deifinição de qual seria a entidade certificadora responsável pelo mesmo. Acredita-se entretanto, que a forma mais sensata de distribuição dos certificados seria feita através da própria organização ou grupo responsável pela manutenção do protocolo. Assim, da mesma forma que em redes como Bitcoin e Ethereum há desenvolvedores dedicados à criação de atualizações e correção de vulnerabilidades, a tarefa de certificação poderia também estar atrelada às atividades dos indivíduos que desempenhariam esse papel na presente proposta. A certificação, como já mencionado na Seção \ref{sec:proposta:definicoes}, seria o único ponto de centralização da rede, porém não afeta de forma alguma a própria forma descentralizada do protocolo, uma vez que a ausência da entidade em um certo ponto no tempo apenas preveniria a entrada de novos prestadores de serviço, não afetando no funcionamento da rede em si e atividades de estabelecimento de contratos e auditorias.

O protocolo também não possui como responsabilidade o controle de funcionamento das alocações de infraestruturas ou funções realizadas internamente por prestadores de serviço, bem como não é seu objetivo criar sistemas de suporte ou redes que tratem especificamente da comunicação com as infraestruturas ou acesso a outros serviços de seus provedores responsáveis. Portanto, a proposta visa estabelecer um ambiente de auxílio no estabelecimento dos contratos e suas auditorias porém não toma nenhuma forma de interação com a implantação dos serviços em si. A mesma lógica aplica-se também para a inserção do módulo de monitoração nas infraestruturas contratadas. 

Por fim, o protocolo não compreende a definição ou implementação do módulo de monitoração, embora faça uso extensivo das funcionalidades do mesmo. Dessa forma, na Seção \ref{sec:proposta:monitoracao} é feita apenas uma descrição de alto nível do que é esperado do módulo em termos de dados de retorno, entrada e realização dos testes, porém as estratégias utilizadas para monitoração, estruturas de dados internas, formas de armazenamento, segurança das chaves e outras características inerentes de implementação não são contempladas por esse trabalho. 

\section{Formas de consenso e ameaças}
\label{sec:ameacas_consenso}

\section{Ameaças gerais}
\label{sec:ameacas_gerai}

Ameaças gerais consistem de um conjunto de possíveis ações que seriam prejudiciais ao protocolo e que teoricamente podem ser efetuadas por qualquer usuário, independente de seu grupo de participação. A primeira e mais facilmente identificável ameaça é a falsificação ou fraude de transações, pois como estas não passam de uma sequência ordenada de dados específicos, basta que um participante encontre uma maneira de criar transações sem utilizar a carteira, replicando assim os padrões de organização dos dados, para que este tenha poder total de gerar transações que normalmente não possuiria permissões ou transações com dados incoerentes ou incorretos. Essa ameaça é felizmente tratada pelo próprio funcionamento do Blockchain em si, uma vez que qualquer usuário pode escrever qualquer transação que não tenha informações verdadeiras e tentar repassá-la, porém quando qualquer outro usuário que seja honesto recebê-la e executar o algoritmo de verificação 
concluirá que a transação é inválida e não a repassará para os próximos nós da rede, evitando assim a possibilidade de que uma transação fraudulenta seja inserida na cadeia. A única possível situação em que uma transação inválida poderia ser inserida na cadeia é se a característica de comprovação da forma de consenso fosse controlada por uma maioria de nós desonestos em concordância de um mesmo objetivo ilegítimo.

%
Ainda no mesmo espectro de ameaças encontra-se a tentativa de utilização de chaves ou certificados falsos, de forma que o usuário anexa tais informações falsas em uma transação, porém novamente a simples verificação pelo nó recebedor já é suficiente para conter tais ataques. Uma forma de vulnerabilidade mais difícil de ser enfrentada é a descoberta de chaves por usuários desonestos, que não pode ser facilmente contornada e que pode desdobrar-se em ataques bastante nocivos através da falsificação bem sucedida de identidade. Entretanto essa não pode ser considerada uma vulnerabilidade do protocolo pois a proteção eficaz de chaves privadas é uma tarefa inerente ao usuário e a segurança contra um possível processo de criptoanálise que venha a resultar no descobrimento de chaves depende do nível de robustez do método de criptografia empregado, que no caso de \ac{ECC} é bastante alto. 

A fase de estabelecimento de contratos não possui um conjunto de vulnerabilidades nocivas, podendo apenas ser alvo de tentativas de falsificação, que como já mencionado nessa seção são ataques facilmente contornados. Isto porque todas as transações dessa fase apenas relacionam dados claros necessários para os contratos e possuem uma ordem de referenciação bastante simples, o que torna a verificação e consequentemente a garantia de segurança do processo muito mais simples uma vez que todas as informações podem ser atestadas ou rejeitadas no processo de validação em si. Contudo, ainda nesse aspecto, um campo em particular poderia apresentar problemas e este é a semente criptografada que deve ser apresentada por prestadores de serviço em qualquer proposta, isso justamente pela incapacidade da validação direta do campo. Assim, nada impediria um prestador de ao invés de seguir o processo correto - escolhendo um número aleatório e o criptografando com a chave pública do módulo - simplesmente publicasse na transação um valor claro qualquer. Esse inconveniente não poderia ser resolvido com assinaturas por exemplo, uma vez que seria necessário a chave privada correspondente para verificá-las e essa é secreta e apenas de posse do módulo. Entretanto, esta não trata-se de fato de uma vulnerabilidade e não prejudica operacionalmente o protocolo, isso por que mesmo que seja publicado um valor qualquer não criptografado, ao ser inserido no módulo o algoritmo de descriptografia será aplicado e um novo número será obtido pois os dados utilizados como entrada serão interpretados como o resultado de um processo criptográfico, mesmo que não o sejam. Assim obtêm-se um novo valor qualquer para a utilização como semente a partir de qualquer valor inserido inicialmente.

O único efeito negativo proveniente de tal ação seria que o provedor não seria capaz de obter a chave pública de votação que possibilita a visualização dos dados dos usuários em uma votação, caso uma viesse a ocorrer. Assim, supõe-se que não haveria nenhuma vantagem real na realização do envio incorreto do que deveria ser a semente criptografada, pois o prestador de serviço apenas estaria abdicando do poder de visualizar os dados a troco de nenhuma recompensa aparente, enquanto o usuário acusador ainda o possuiria. O mesmo raciocínio naturalmente se aplica às sementes fornecidas pelos clientes nas convocações de votação.

Ao tratar da fase de auditoria, existe um conjunto relativamente maior de possíveis ameaças ao funcionamento do protocolo devido principalmente ao uso de criptografia, sensibilidade dos dados e interações mais restritivas entre os participantes. Ao retomar o conceito de pontos de Schelling para a obtenção de conclusões coletivas, não deve ser permitida a troca de informações entre os participantes, pois a comunicação entre estes permitiria o surgimento de conluios

%
O protocolo, entretanto, não possui a capacidade de evitar a comunicação deliberada de dados de votação entre os participantes, isso porque embora os dados sejam criptografados de diferentes formas em períodos distintos da votação, deve haver inevitavelmente uma forma de visualizá-los para que seja possível gerar o veredito coletivo apropriado. Atrelado a este fator, como o módulo de monitoração responsável por gerar e armazenar tais dados funciona de maneira \textit{standalone} e assíncrona, a única porta de saída de informações existente é através do próprio usuário dono que as requisita. Assim, não havendo garantias da idoneidade das intenções deste usuário quanto aos seus dados de monitoração, esse pode seguir os passos descritos na Seção \ref{fig:votacao_detalhada} para iniciar uma votação, que o módulo naturalmente assume ser honesta. Entretanto, ao obter os dados criptografados e a chave pública que os revela, o usuário não inicia de fato uma votação na cadeia, visualiza seus próprios dados e os compartilha com outros participantes ou os utiliza de outra forma não permitida.

%
A partir dessa análise, usuários desonestos podem de forma bastante simples ter acesso aos seus próprios dados e dar diversos fins para essa informação, porém não é possível que um usuário desonesto tenha de alguma maneira acesso a dados de usuários terceiros honestos no que toca a segurança do protocolo em si. Tal constatação pode ser concebida pois no caso de uma convocação de votação honesta qualquer usuário, honesto ou não apenas terá acesso aos dados claros no final da votação. No caso do envio de um voto honesto, também não há como visualizar os dados uma vez que estes estão criptografados pela chave de votação, conhecida apenas pelo usuário acusador e provedor acusado.

%
Por si só, a divulgação dos resultados de monitoração não representa um problema. Complicações relacionadas à integridade das votações originadas de revelações apenas surgiriam se os usuários recebedores de informações as utilizassem como referencial para criar novos dados falsos ou adulterar dados já existentes, buscando com isso dirigir a votação a um estado mais lucrativo. Contudo, o protocolo contém mecanismos que evitam tanto a forja quanto a emulação de dados mesmo com a presença de revelações e estes são relacionados nas Seções \ref{sec:ameacas_clientes} e \ref{sec:ameacas_prestadores} juntamente a outras ameaças individuais relacionadas a cada um dos grupos de usuários.


%A proposta faz uso parcial do conceito dos pontos de Schelling na forma de votações para averiguar a existência ou não de quebras de contrato. Através disso e segundo a teoria dos jogos, para que o conceito funcione corretamente não deve haver comunicação entre os participantes a respeito da forma, conteúdo ou outras características dos dados que devem ser compartilhados para a obtenção do veredito. Contudo, não há como impedir a visualização dos dados por parte por parte de seus donos, ou seja os usuários responsáveis pelos módulos que geram os dados.

\section{Ameaças oriundas dos clientes}
\label{sec:ameacas_clientes}

\section{Ameaças oriundas dos prestadores de serviço}
\label{sec:ameacas_prestadores}

\section{Considerações finais}
\label{sec:consideracoes_ameacas}


%Um problema inerente aos contratos de serviços relacionados às nuvens computacionais para microsserviços, principalmente quando fornecidos a usuários nas bordas da Internet que não possuem controle sobre dados relacionados ao serviço \cite{nuvem_sla:edge_computing}, trata-se da falta de garantia de cumprimento do acordado por parte do provedor.

%
% Por meio da proposta apresentada no Capítulo \ref{ch:proposta} intende-se resolver o problema da inexistência de um ambiente concreto para elaboração de contratos \ac{IaaS} para microsserviços e da impossibilidade de garantir a veracidade de acusações de violação ou provar a violação em casos verdadeiros. A solução proposta utiliza primordialmente a tecnologia de \textit{Blockchain} para criar um protocolo que permita a elaboração e auditoria dos contratos em questão. As próximas etapas do desenvolvimento do protocolo envolvem a definição da hierarquia de transações e a definição técnica de cada uma delas, o funcionamento detalhado da carteira, a especificação da distribuição de pontos de reputação e outros aspectos importantes.