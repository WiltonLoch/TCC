%TODOS OS PACKAGE DEVEM SER COLOCADOS AKI
\usepackage[brazil,brazilian,portuges]{babel}
\usepackage[T1]{fontenc}
%\usepackage[brazilian]{babel}
\usepackage{mathptmx}

\usepackage{amsmath,listings,indentfirst,url,hyperref}
%\usepackage{amsmath,listings,indentfirst}
\usepackage{paralist}
\usepackage[all]{hypcap}
\usepackage[alf]{abntex2cite}
\usepackage[portuguese]{nomencl}
\usepackage{longtable}
\usepackage{fancyvrb}
\usepackage{color}
\usepackage{rotating}
\usepackage{multirow}
\usepackage{enumitem}
\usepackage[portuguese, ruled, linesnumbered]{algorithm2e}
\usepackage{latexsym}
\usepackage[table]{xcolor}
\usepackage[utf8]{inputenc}
\usepackage[section]{placeins}
\usepackage{graphicx}
\usepackage{caption}
%%resize table op1
\usepackage{adjustbox}
%\usepackage{slashbox,booktabs,amsmath}
\usepackage{diagbox}
\usepackage{enumitem}
\usepackage{booktabs}
\usepackage{quoting}
\usepackage{setspace}
\usepackage[table]{xcolor}

\usepackage{subcaption} 
%só colocar 
%\adjustbox{max height=\dimexpr\textheight-5.5cm\relax,
%           max width=\textwidth}{
%antes do tabular 
%lembrar de fechar o { no \end{tabular} }


% pacotes de fontes! doi ter que usar times para evitar problemas :(
%\usepackage{fontspec}
%\usepackage{xunicode} 
%\defaultfontfeatures{Mapping=tex-text} 
%\setromanfont{Garamond}
%\setsansfont{Gill Sans}
%\setmonofont{Courier New}

% Listings
\lstset{
    inputencoding=utf8,
    basicstyle=\scriptsize,
    frame=single,
	tabsize=4,
	captionpos=b,
	breaklines=true,
	numbers=left,
}

% Não sei porque mas o LaTeX insiste em avançar a margem nas citações.
% Felizmente há o comando \sloppy, que diz para aumentar o espaçamento
% entre as palavras, visando /sempre/ respeitar as margens. Fica feio
% em alguns parágrafos, mas é a vida...
\sloppy

% da um trato nos floats
\renewcommand{\topfraction}{.85}
\renewcommand{\bottomfraction}{.7}
\renewcommand{\textfraction}{.15}
\renewcommand{\floatpagefraction}{.66}
\renewcommand{\dbltopfraction}{.66}
\renewcommand{\dblfloatpagefraction}{.66}
\setcounter{topnumber}{9}
\setcounter{bottomnumber}{9}
\setcounter{totalnumber}{20}
\setcounter{dbltopnumber}{9}

%\addto\captionsportuges{
%  \renewcommand{\tablename}
%    {Quadro}}

%\addto\captionsportuges{
%  \renewcommand{\listtablename}
%    {Lista de Quadros}}

\renewcommand{\lstlistingname}{C\'odigo}
\def\listofsymbols{\def\addsymbol #1 #2{#1  & \hspace{0.5in} #2 \\ } 

\begin{tabular}{l l}

% A
% B
% C
% D
% E
% F
% G
% H

%I
    
    \addsymbol IaaS {\textit{Infrastructure as a Service}}
% J
% K
% L
% M
% N
% O
% P
% Q
% R
% S
    \addsymbol SC {Contratos Inteligentes(\textit{Smart Contracts})}
% T
% U
% V
% W
% X
% Y
% Z

\end{tabular}

 \clearpage}

% Verbatim não precisa de espaçamento duplo e nem de fontes tão grandes
\RecustomVerbatimEnvironment
	{Verbatim}%
	{Verbatim}%
	{baselinestretch=1,	fontsize=\relsize{-1}}



%%%%%%%%%%%%%%count enumarete start in zero
\usepackage{enumitem}
\setlist[enumerate,1]{start=1} % only outer nesting level

%%%%%%%%cite inside enumerate
\makeatletter
\newcommand{\mylabel}[2]{#2\def\@currentlabel{#2}\label{#1}}
\makeatother

