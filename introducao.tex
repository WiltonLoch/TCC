\chapter{Introdução}
\label{ch:intro}

Desde os primórdios da computação a parte majoritária das aplicações vem sendo desenvolvidas de maneira monolítica, ou seja, estruturadas em grandes blocos de código que juntos em um arquivo ou projeto formam o sistema em si \cite{microsservicos:definicao_microsservicos}.
%
Nesse modelo, os componentes do \textit{software} são fortemente conectados e a modularização dos elementos pode ser feita apenas pela divisão de arquivos dentro de um mesmo projeto ou ambiente de trabalho.
%
As aplicações são desenvolvidas puramente através da construção incremental sobre tal conjunto de arquivos, nos quais diferentes desenvolvedores podem trabalhar em uníssono sobre partes diferentes do mesmo monólito integrado.

%A evolução do \textit{software} bem como o acompanhamento do progresso é feito através de sistemas de versionamento consolidados no mercado, contribuindo também para a divisão do trabalho entre diferentes colaboradores.
%
%Dentre outras diversas características do modelo monolítico, inerentes à sua estrutura em si, temos o conjunto que forma uma das arquiteturas clássicas no desenvolvimento de sistemas.
%

%
De modo geral, esse modelo de arquitetura funcionou bem por muito tempo e até então tem atendido à boa parte das necessidades de desenvolvedores e empresas com aplicações que não possuem grande porte, tanto do ponto de vista de código quanto de uso.
%
Porém, com a recorrente facilitação e popularização do acesso à tecnologia é natural que haja um aumento significativo na quantidade de usuários que diariamente conectam-se às aplicações e realizam contínuas requisições, bem como uso de dados.
%
Consequentemente, a necessidade de melhorias e de disponibilidade do sistema aos usuários aumenta, expandindo consigo a quantidade de código e de funcionalidades que os mesmos devem prover.
%
Assim, problemas já existentes no modelo arquitetural monolítico se tornaram ainda mais evidentes e prejudiciais para o processo de desenvolvimento.
%
Fatores como controle e compartilhamento de código, distribuição de versões, correção de \textit{bugs} e falhas acabam tornando-se tarefas penosas em grandes aplicações, prejudicando assim a eficiência e a competitividade das empresas que se ativeram ao modelo \cite{microsservicos:uber}.

%
Para desviar desses problemas eminentes não seria suficiente apenas aumentar a quantidade de \textit{hardware} à disposição das aplicações; era necessária uma reformulação da maneira de construí-las em si.
%
Essa alternativa explorada por empresas como Uber \cite{microsservicos:uber}, Netflix \cite{microsservicos:netflix} e diversas outras \cite{microsservicos:empresas} para mitigar os problemas decorrentes do monólito foi rearquitetar os sistemas para que funcionassem através de um conjunto de microsserviços.
%
Estes, ao comunicarem-se entre si, realizam as funções do sistema de maneira independente e distribuída. Desta forma, uma parte central da aplicação interage com diversos microsserviços remotamente - utilizando de comunicação via rede - sem se preocupar com a execução da tarefa de cada um deles, porém utilizando seus serviços como partes constituintes da execução como um todo.

%
Os microsserviços, por tratarem-se basicamente das partes fundamentais de um sistema distribuído, podem ser facilmente alocados em máquinas distintas sobre a rede.
%
Porém, ao tratar demandas de dados e requisições elevadas torna-se necessário o uso de \textit{hardware} que dê suporte para tal da mesma forma que outras medidas para garantir características como confiabilidade e capacidade de sobrevivência das alocações.
%
Entretanto, o investimento necessário para a compra e manutenção de \textit{hardware}, bem como a garantia do funcionamento adequado do mesmo para a aplicação podem não ser esforços atrativos para muitas companhias que utilizam microsserviços.
%
Torna-se então genuinamente interessante para tais empresas contratar provedores que forneçam a infraestrutura necessária como um serviço. Uma escolha interessante nesse meio são as tradicionais alocações de \ac{IaaS}, onde o provedor supre uma infraestrutura que deve ser administrada pelo usuário, sendo este responsável pela implantação e execução do serviço. Outra possibilidade é a alocação de \ac{FaaS}, cujas quais atuam como microsserviços executando tarefas específicas. Um exemplo de uso desta forma de serviço é o redimensionamento de imagens, necessário em sites com conteúdo em tempo real, \textit{stream} de vídeo, jogos, \textit{etc}. Para tal, uma ou mais unidades \ac{FaaS} alocadas executam simplesmente a tarefa de redimensionar uma certa imagem e retorná-la ao requisitante mediante a ocorrência da chamada. 

%
Há, entretanto, um problema inerente aos contratos de serviços relacionados às nuvens computacionais, principalmente quando fornecidos a usuários nas bordas da Internet que não possuem controle sobre dados relacionados ao serviço \cite{nuvem_sla:edge_computing}, que trata-se da falta de garantia de cumprimento do acordado por parte do provedor, de modo que este pode fornecer menos recursos do que o previamente estabelecido em contrato, injustiçando assim seus clientes \cite{nuvem_sla:violacaoSLA}. Casos comuns de quebra de contrato no sentido dos provedores envolvem o não fornecimento de recursos como o prometido, principalmente tratando-se de poder computacional, quantidade de memória primária disponível, largura de banda e tempo de resposta.

%Tais quebras de contrato derivam de fatores como alocação de \textit{hardware} inferior, compartilhamento das unidades físicas por diversos usuários ou simplesmente má intenção dos responsáveis por fornecer o serviço. 
%
Pelo outro lado, clientes que contratam o serviço podem agir de maneira desonesta, acusando os provedores de negligência no fornecimento do serviço e buscando com isso algum tipo de vantagem mercadológica ou jurídica \cite{nuvem_sla:clientes_desonestos, nuvem_sla:clientes_desonestos2}.
%
Portanto, torna-se uma tarefa complexa aferir e julgar corretamente qual das partes envolvidas no contrato foi lesada e qual buscou um benefício ilícito em face de uma disputa contratual.

Para tratar esse problema o presente trabalho propõe um protocolo, que segundo a definição de Tanembaum de \citeyear{nuvem_sla:tanenbaum}, é um acordo entre as partes que se comunicam, estabelecendo como se dará a comunicação. Portanto, o protocolo proposto formalizará como as mensagens serão trocadas entre os clientes e os provedores, a ordem das mesmas, bem como seus formatos estruturais.
%
%
O protocolo é baseado em Blockchain que busca garantir transparência para as transações e requisições entre provedores e clientes para alocação de microsserviços em ambientes \ac{IaaS}. É proposto também o desenvolvimento de um algoritmo distribuído sobre a cadeia que realize a conferência de uma reclamação, recompensando participantes que forneçam informações corretas e penalizando os que não o façam.
%
O uso de Blockchain é justificado por sua habilidade de fomentar um ambiente distribuído com fácil verificação da existência e corretude de dados em um certo ponto no tempo. Tal característica torna possível auditar trocas de informação entre clientes e provedores, servindo como ponto de partida para a tomada de decisão a respeito de possíveis violações através do algoritmo proposto. Este, por sua vez, é centrado em uma votação assíncrona e distribuída, que utiliza como voto um histórico de monitoração mutuamente realizado por usuários e provedores, para inferir ocorrências de quebra de contrato resultantes da negligência do fornecedor.
%
O trabalho então, possibilita com uma porcentagem de certeza alta e previsível afirmar se houve ou não violações de contratos - desde que firmados dentro da rede - enquanto faz uso das vantagens promovidas por tecnologias de armazenamento distribuído, troca de mensagens e votação.

%\clearpage
%------------------------------------------------
\section{Objetivos}

Este trabalho tem como objetivo o desenvolvimento de um protocolo, baseado em
Blockchain para auxiliar a elaboração e auditoria de contratos entre usuários e provedores de IaaS para microsserviços.

\vskip 0.5cm
\noindent\textbf{Objetivos específicos}: 
\begin{itemize}
    \item Realizar uma revisão bibliográfica sobre o gerenciamento de microsserviços provisionados em provedores de nuvem e na borda da Internet.
    \item Realizar uma revisão bibliográfica sobre o estabelecimento de contratos usando Blockchain em cenários competitivos e egoístas.
    \item Identificar os elementos comuns no estabelecimento de contratos entre provedores de microsserviços e usuários.
    \item Especificar e desenvolver um protocolo para tal finalidade.
    \item Realizar um estudo sobre as ameaças ao protocolo com intuito de quantificar o benefício da proposta.
\end{itemize}


\section{Estrutura do Trabalho}

Este trabalho é estruturado em quatro capítulos. No Capítulo \ref{ch:fundamentos}, trata-se a fundamentação teórica do presente trabalho, revisando conceitos relacionados à microsserviços e \textit{Blockchains} e suas características principais, bem como levantados trabalhos relacionados e comparados com a presente proposta. O Capítulo \ref{ch:proposta} concentra o desenvolvimento do presente trabalho, o cenário de execução, o conjunto de transações e definições técnicas gerais do protocolo proposto. No capítulo \ref{ch:ameaças} é apresentado um levantamento sobre as possíveis ameaças ao protocolo e como sua arquitetura as combate e por fim, no capítulo \ref{ch:conclusao} são apresentadas as considerações finais do presente trabalho..

% TODO: atualizar após o acréscimo do novo capítulo.