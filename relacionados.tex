\chapter{Trabalhos Relacionados}
\label{ch:relacionados}

Estudos envolvendo as nuvens computacionais e a aplicação dos métodos multicritérios vêm sendo conduzidos há vários anos. Esta Seção apresenta os trabalhos que analisam o desempenho dos métodos multicritérios.
%
O trabalho de \citeonline{whaiduzzaman2014cloud} teve como foco identificar diversos métodos multicritérios utilizados em nuvens computacionais e apresentar uma taxonomia de AMD. Diversos exemplos de aplicações que utilizam os métodos foram apresentados, demonstrando a grande importância e efeito que o AMD proporciona em diversos cenários. 

Os autores \citeonline{Multi} analisaram plataformas de computação hibridas (\textit{i,e,.} compostas por \textit{clusters} e nuvens computacionais), definindo os principais aspectos das tecnologias e introduzem as diferenças entre algoritmos de alocação globais e locais, avaliando-os através de simulações de eventos discretos. Os algoritmos utilizados foram o ACA (do inglês, \textit{Allocated Cluster Algorithm}) , CCA (do inglês, \textit{Cloud Cluster Algorithm}) e DCA (do inglês, \textit{Dedicated Cluster Algorithm}), onde possíveis modificações podem ser realizadas para a avaliação de critérios adicionais. 

\citeonline{ristova2011ahp} desenvolveram uma revisão sobre a aplicação do método AHP nos ciclos de gerenciamento de produto, provendo decisões com maiores informações. A maior dificuldade encontrada nos processos de decisão é a impossibilidade de ranqueamento das alternativas somente com o resultado da análise financeira. Os custos de ganho de escalabilidade e agilidade de resposta as mudanças de negócios são custos que devem ser inclusos por apresentarem um grande impacto no resultado final. Os critérios definidos para melhorar a análise do gerenciamento de produto a escalabilidade e fatores de riscos econômicos e tecnológicos e concluem que a utilização do AHP melhora trás melhor acurácia nos resultados quando comparado aos métodos padrões de análise financeira.

Por sua vez, \citeonline{yazir2010dynamic} focou seus estudos na decomposição do gerenciamento de recursos de uma nuvem computacional em tarefas independentes, as quais são executadas por agentes autônomos acoplados em máquinas físicas de um \textit{data center}. Estes agentes realizam configurações em paralelo através da utilização do método PROMETHEE. Esta abordagem garante aos provedores escalabilidade, flexibilidade e viabilidade através da troca rápida dos pesos dos critérios e da possibilidade de se adicionar ou remover um critério ao invés de mudar a configuração da nuvem.

O trabalho de \citeonline{martens2012decision} tem o objetivo de determinar a seleção de serviços oferecidos por diferentes provedores de computação em nuvem através do gerenciamento de riscos e aplicação do método AHP. Para o escopo do problema, os riscos (\textit{e,g,.} serviços de TI, custos de manutenção) são considerados e modelados de acordo com os três pilares da segurança: integridade, confidencialidade e disponibilidade. A aplicação deste modelo é a habilidade de identificar o serviço mais adequado para uma organização com mais precisão.

O trabalho de \citeonline{ergu2013analytic} possui foco na aplicação do AHP na nuvem computacional, propondo um modelo para a alocação de recursos,  considerando a comparação paritária de acordo com a largura de banda, tempo, custos e confiabilidade das tarefas executadas. O modelo apresentado busca melhorar a razão de consistência  para aqueles casos que a razão já é menor que 0.1, uma vez que os recursos são alocados de acordo com os pesos atribuídos a tarefa requisitada e dos recursos computacionais utilizados para cumprir esta tarefa.

No geral, os trabalhos de análise de desempenho dos métodos multicritérios em nuvens computacionais buscam melhorar o desempenho dos provedores da nuvem (\textit{e,g,.} solucionar o problema de VNE), sendo poucos trabalhos que analisam a visão do usuário da nuvem.

A Tabela~\ref{table:relacionados} resume os trabalhos discutidos, indicando o método multicritério utilizado e o foco do trabalho.

\begin{table}[ht]
    \centering
    \resizebox{\textwidth}{!}{%
        \begin{tabular}{|l|p{0.30\linewidth}|p{0.45\linewidth}|}
        \hline
        \textbf{Trabalho} & \textbf{Método utilizado} & \textbf{Foco do trabalho} \\ \hline
        
            \cite{whaiduzzaman2014cloud} &  AHP, ANP, PROMETHEE, ELECTRE, MAUT, dentre outros. & Sintetizar os métodos AMD utilizados em nuvens computacionais.\\ \hline
        
        
            \cite{Multi} & ACA, CCA, DCA   & Propor algoritmos de alocação de recursos multicritérios que tratem de forma eficiente plataforma híbridas. \\ \hline
            
            \cite{ristova2011ahp} & AHP & Identificar o ganho de escala na agilidade nos processos de decisão. \\ \hline
        
            \cite{yazir2010dynamic} & PROMETHEE &  Propor uma nova abordagem para os recursos dinâmicos de gerenciamento de nuvens computacionais.\\ \hline
        
            \cite{martens2012decision} & AHP & Apresentar um modelo de decisão para dar suporte aos serviços de computação em nuvem. \\ \hline
        
            \cite{ergu2013analytic} & AHP & Propor um modelo para a alocação de recursos nas nuvens computacionais.\\ \hline
        
        \end{tabular}%
    }
    \caption{Resumo dos trabalhos relacionados}
    Fonte: o autor.
    \label{table:relacionados}
\end{table}